
\newpage
\begin{center}
    \huge \textbf{\textcolor{orange}{HydroBot Website Overview}} \\[0.5cm]
\end{center}

\section{\large \textbf{Dive Into Discovery}}

\begin{figure}[h!]
  \centering
  \begin{minipage}{0.6\textwidth}
    \centering
    \includegraphics[width=\textwidth]{img/WEBSITE/WEBSITE.png}
  \end{minipage}%
    \begin{minipage}{0.4\textwidth}
        \centering
        \qrcode[height=5cm]{https://mindcraftlabs.github.io/Dive-Into-Discovery/}
    \end{minipage}
\end{figure}

\begin{center}
    Scan to Visit Our Website!
\end{center}
    
Our HydroBot project is presented on an interactive website, built with HTML, CSS, and JavaScript for a seamless user experience. Key features include:

- \textbf{Responsive Design}: The website adjusts to various devices, ensuring an optimal viewing experience on desktops, tablets, and smartphones.
  
- \textbf{Interactive Interface}: Users can explore HydroBot through interactive buttons, sliders, and animations for an engaging experience.

- \textbf{Detailed Sections}: The site is divided into clear, informative sections that explain the project in-depth, including its overview, components, and specialized areas, allowing users to navigate and understand the information with ease.

- \textbf{Smooth Navigation}: Menus and navigation bars make it easy to find info on HydroBot's design, functionality, and technology.

This user-centric website makes it easy for visitors to learn everything about HydroBot through an accessible, well-structured, and interactive design.

\begin{center}
  \includegraphics[width=0.6\textwidth]{img/WEBSITE/WEBSITE features.png}
\end{center}

\begin{center}
  \includegraphics[width=0.6\textwidth]{img/WEBSITE/AI website.png}
\end{center}





\newpage
\begin{center}
    \huge \textbf{\textcolor{orange}{OPEN SOURCE}} \\[0.5cm]
 
\end{center}
\section{\large \textbf{The Open-Source Nature of HydroBot}}

\subsection{{What is Open Source?}}

Open source refers to software, designs, or technologies that are made available to the public for free, allowing anyone to view, modify, and distribute it.
\vspace{1cm}
\begin{figure}[h!]
    \centering % Centers the figure
    \begin{minipage}{0.35\textwidth}
        \centering
        \includegraphics[width=\textwidth]{img/innovation project pic/open-source.png}
    \end{minipage}
\end{figure}

\vspace{1cm}
\subsection{{Why Open Source?}}
Choosing to make HydroBot open-source is a strategic decision that aligns with our mission to create a global and sustainable solution to ocean pollution.  Here’s why we believe open-source is the best path forward:

\begin{itemize}
    \item \textbf{\textcolor{orange}{Collaboration}}: The open-source community works together to improve the project and share ideas.
    \item \textbf{\textcolor{orange}{Transparency}}: Open-source projects are transparent in terms of development, licensing, and usage.
    \item \textbf{\textcolor{orange}{Community-driven}}: it's powered by a global network of contributors who collaborate, discuss, and innovate together.
    \item \textbf{\textcolor{orange}{Free and Open Access}}: Open-source software is freely available for anyone to use, modify, and distribute.
    \item \textbf{\textcolor{orange}{Sustainability}}: Open-source projects rely on community support for adaptability and longevity.
    \item \textbf{\textcolor{orange}{Inclusivity}}: Open-source fosters inclusivity by allowing anyone,to contribute to the project.
\end{itemize}

\newpage
\begin{center}
    \huge \textbf{\textcolor{orange}{HydroBot App}} \\[0.5cm]
 
\end{center}
\section{\large \textbf{HydroBot App}}
The HydroBot mobile app offers full control and monitoring of the robot, providing users with a seamless way to interact with its features. Designed for convenience, the app allows users to track system performance, manually control movements, or enable AI-driven automation.

\subsubsection{Main Features}
\begin{center}
  \includegraphics[width=0.6\textwidth]{img/WEBSITE/hydrobot app.png}
\end{center}

\vspace{1cm}
\begin{minipage}{0.6\textwidth}
    - \textbf{\textcolor{orange}{Connect via IP Address}}: Users can connect to HydroBot by entering the Raspberry Pi's IP address, ensuring secure and direct communication.

    - \textbf{\textcolor{orange}{Robot Control}}: The app provides manual controls for guiding and driving the robot remotely.

    - \textbf{\textcolor{orange}{Garbage Auto-Detect}}: Using AI, the app can automatically detect and respond to garbage in its path.

    - \textbf{\textcolor{orange}{AI-Driven Mode}}: Users can switch to AI mode, allowing the robot to navigate and operate autonomously based on real-time data.

    Additionally, the app tracks essential metrics such as battery level and CPU temperature, ensuring users stay informed about the robot’s status at all times.
\end{minipage}
\hfill
\begin{minipage}{0.35\textwidth}
    \centering
    \includegraphics[width=0.7\textwidth]{img/WEBSITE/mobile app.png}
\end{minipage}


\newpage
\begin{center}
    \huge \textbf{\textcolor{orange}{FEEDBACK}} \\[0.5cm]
 
\end{center}
\section{\large \textbf{FeedBack}}
In order to gather meaningful feedback for our project, we created a Google Form that was distributed to a diverse group of people. To ensure that everyone could understand and participate, we designed the questions in Moroccan dialect, which made the form more accessible.
\subsection{\textbf{Google Form}}


The form contained several questions related to the ocean, its pollution, and marine life. Some of the key questions included:

\begin{itemize}
    \item How do you perceive ocean pollution?
    \item Who do you think is responsible for ocean pollution?
    \item Do you believe robots can actually clean the oceans and remove debris?
\end{itemize}

These questions were designed to engage the participants and gather their opinions on ocean pollution, as well as their thoughts on the potential role of technology in addressing the issue.

\begin{figure}[h!]
    \centering
    % First row with two images side by side
    \begin{minipage}{0.45\textwidth}
        \centering
        \includegraphics[width=1.3\linewidth]{img/innovation project pic/Google Form/bar chart 1.png}
        \caption{What is your level of expertise in the field of ocean studies?}
    \end{minipage}
    \hfill
    \begin{minipage}{0.45\textwidth}
        \centering
        \includegraphics[width=1\linewidth]{img/innovation project pic/Google Form/pie chart 1.png}
        \caption{Do you think ocean pollution is increasing over time?}
    \end{minipage}
    
    \vspace{1cm} % Space between rows
    
    % Second row with two images side by side
    \begin{minipage}{0.45\textwidth}
        \centering
        \includegraphics[width=\linewidth]{img/innovation project pic/Google Form/pie chart 3.png}
        \caption{Do you know of any devices or technologies that clean oceans?}
    \end{minipage}
    \hfill
    \begin{minipage}{0.45\textwidth}
        \centering
        \includegraphics[width=1.2\linewidth]{img/innovation project pic/Google Form/bar chart 2.png}
        \caption{How important do you think a robot's ocean-cleaning capability is?}
    \end{minipage}
\end{figure}


\begin{figure}[h!]
    \centering
    % First row with two images side by side
    \begin{minipage}{0.45\textwidth}
        \centering
        \includegraphics[width=1.3\linewidth]{img/innovation project pic/Google Form/horizontal bar chart.png}
        \caption{In your opinion, what is the main cause of ocean pollution?}
    \end{minipage}
    \hfill
    \begin{minipage}{0.45\textwidth}
        \centering
        \includegraphics[width=1\linewidth]{img/innovation project pic/Google Form/pie chart 2.png}
        \caption{Do you think robots can actually solve the ocean pollution problem?}
    \end{minipage}
    
    \vspace{1cm} % Space between rows

\end{figure}
 \vspace{3cm}
This Google Form helped improve our robot by gathering feedback on ocean pollution and robot capabilities. The responses guided us in refining the robot's design and focusing on features that are most important for cleaning the ocean effectively.



\subsection{Talking with an Expert about Ocean Pollution}
During our discussion with an oceanography expert,\textbf{\textcolor{orange}{Naoufal Elouardi}}, we explored the critical issue of ocean pollution and its consequences. The expert highlighted how pollution affects marine ecosystems and human endeavors alike.

\subsubsection{Challenges for Ocean Cleanup Efforts}
The expert highlighted that pollution poses numerous obstacles to cleaning efforts, including:

\begin{itemize}
    \item \textbf{Water Contamination:} Plastics, chemicals, and oil spills degrade water quality, making cleanup operations more difficult.
    \item \textbf{Microplastic Detection:} Tiny plastic particles are challenging to detect and remove efficiently.
    \item \textbf{Artificial Light Disruption:} Excessive human-generated light affects marine organisms and interferes with underwater monitoring.
    \item \textbf{Equipment Durability:} Corrosive pollutants can damage cleanup devices, reducing their operational lifespan.
\end{itemize}

\newpage
\subsection{Meet the expert}
\begin{minipage}{0.6\textwidth}
    \centering
    \includegraphics[width=0.8\textwidth]{img/profil pics/Naoufal OURDI.jpg}
\end{minipage}
\hfill
\begin{minipage}{0.35\textwidth}
    \centering
    \includegraphics[width=1.4\textwidth]{img/profil pics/Naoufal OURDI profil.jpg}
\end{minipage}
\subsubsection{Biography}

\textbf{Naoufal Ouardi} is a Marine Architect with 15 years of experience in port construction and the development of marine systems. Throughout his career, he has been involved in numerous projects, contributing to the advancement of maritime infrastructure and innovative marine technologies.

After we shared our \textbf{HydroBot} project with him, he provided valuable recommendations to enhance its capabilities, including:

\begin{itemize}
    \item Expanding the operational range to improve coverage and efficiency.
    \item Integrating an AI model for autonomous navigation and decision-making.
    \item Adding a solar panel to increase endurance and energy autonomy.
\end{itemize}

His expertise and insights have significantly contributed to the improvement of our project, making it more efficient and sustainable.
\begin{figure}[h]
  \centering
  \includegraphics[width=1\textwidth]{img/innovation project pic/meeting expert.png}
  \caption{Evidence of our meeting with the expert.}
  \label{fig:meeting_expert}
\end{figure}


\newpage
\subsection{Incorporation of Feedback}
Thanks to the feedback collected through the Google Form, we were able to make significant improvements to the robot. Key updates include:

\begin{itemize}
    \item The addition of \textbf{\textcolor{orange}{solar panels}} to improve energy efficiency and extend operational time, addressing concerns about sustainable energy sources.
    \item Integration of a \textbf{\textcolor{orange}{camera}} for better navigation and monitoring of the cleaning process, based on feedback about enhancing the robot's effectiveness in different environments.
    \item Prioritized features like improving the robot's waste collection system.
    \item Provided insights on user expectations and potential challenges.
  
\end{itemize}
\begin{figure}[h!]
    \centering
    % First row with two images side by side
    \begin{minipage}{0.45\textwidth}
        \centering
        \includegraphics[width=\linewidth]{img/innovation project pic/HYDROBOT/HydroBot2.png}
       
    \end{minipage}
    \hfill
    \begin{minipage}{0.45\textwidth}
        \centering
        \includegraphics[width=\linewidth]{img/innovation project pic/HYDROBOT/HydroBot3.png}
       
    \end{minipage}
\end{figure}




\newpage
\begin{center}
    \huge \textbf{\textcolor{orange}{IMPACT}} \\[0.5cm]
 
\end{center}
\section{\large \textbf{Impact}}
Our Hydrobot presents a sustainable and innovative approach to tackling ocean pollution.The potential impact of our solution extends beyond environmental benefits; it also contributes to cleaner resources for marine biodiversity, safer waters for human activities, and technological advancements in environmental conservation.

\subsection{Alignment with Sustainable Development Goals}
Our project aligns with several United Nations Sustainable Development Goals (SDGs). Below is a breakdown of each goal, its relevance, and how our Hydrobot contributes to it.


\renewcommand{\arraystretch}{0.8} 
\setlength{\tabcolsep}{5pt} % Adjust space between columns to make them smaller
\begin{longtable}{|m{3.5cm}|m{13cm}|} % Increase column widths

    \hline
    \textbf{Goal} & \textbf{Impact of the Hydrobot} \\
    \hline
    \centering \includegraphics[width=3cm]{img/innovation project pic/UN goals pic/goal6.PNG} & \large \textbf{Goal 6: Clean Water and Sanitation} \\ 
    & \vspace{-0,2pt}  The HydroBot  helps improve water quality and support the sustainable management of water resources through the removal of pollutants like microplastics and debris. \\
    \hline
    \centering \includegraphics[width=3cm]{img/innovation project pic/UN goals pic/goal7.PNG} &  \large\textbf{Goal 7: Affordable and Clean Energy} \\ 
    & \vspace{-0,2pt} HydroBot uses solar panels, a renewable energy source, to power its operations. This reduces reliance on non-renewable energy and promotes sustainable energy solutions.  \\
    \hline
    \centering \includegraphics[width=3cm]{img/innovation project pic/UN goals pic/goal9.PNG} &  \large\textbf{Goal 9: Industry, Innovation, and Infrastructure} \\ 
    & \vspace{-0,2pt} The project promotes innovative engineering solutions, demonstrating how technology can be used to address global environmental challenges. \\
    \hline
    \centering \includegraphics[width=3cm]{img/innovation project pic/UN goals pic/goal11.PNG} & \large\textbf{Goal 11: Sustainable Cities and Communities} \\ 
    & \vspace{-0,2pt} HydroBot reduces ocean pollution, improves environmental quality, and helps create cleaner, more resilient coastal urban areas. \\
    \hline
    \centering \includegraphics[width=3cm]{img/innovation project pic/UN goals pic/goal13.PNG} &  \large\textbf{Goal 13: Climate Action} \\ 
    & \vspace{-0,2pt} Ocean pollution directly affects climate change; our Hydrobot helps mitigate this by reducing marine contaminants and preserving ecosystems. \\
    \hline
    \centering \includegraphics[width=3cm]{img/innovation project pic/UN goals pic/goal14.PNG} &  \large\textbf{Goal 14: Life Below Water} \\ 
    & \vspace{-0,2pt} Protecting and restoring marine ecosystems is the core mission of our project, directly addressing ocean pollution and ensuring healthier waters for marine life. \\
    \hline

\end{longtable}


\newpage
\begin{center}
    \huge \textbf{\textcolor{orange}{PROGRAMMING TECHNIQUES AND 3D DESIGN TOOLS}} \\[0.5cm]
\end{center}

\section{\large \textbf{Programming techniques and 3D design tools}}
\subsection{\large \textbf{Design and programming}}

\begin{figure}[htbp]
    \centering
    \begin{minipage}{0.2\textwidth}
        \centering
        \includegraphics[width=\textwidth]{img/logo pic/design&programming/Canva.png}
        \vspace{0.3cm}
        \textbf{Canva}\\
        \small{Used for creating visually appealing graphics and designs.}
    \end{minipage}%
    \hfill
    \begin{minipage}{0.2\textwidth}
        \centering
        \includegraphics[width=\textwidth]{img/logo pic/design&programming/VScode.png}
        \vspace{0.3cm}
        \textbf{VSCode}\\
        \small{Main programming environment for coding.}
    \end{minipage}%
    \hfill
    \begin{minipage}{0.2\textwidth}
        \centering
        \includegraphics[width=\textwidth]{img/logo pic/design&programming/Thonny.png}
        \vspace{0.3cm}
        \textbf{Thonny}\\
        \small{Main programming environment used for coding.}
    \end{minipage}%
    \hfill
    \begin{minipage}{0.2\textwidth}
        \centering
        \includegraphics[width=\textwidth]{img/logo pic/design&programming/premiere pro.png}
        \vspace{0.3cm}
        \textbf{Premier Pro}\\
        \small{Used for video editing.}
    \end{minipage}%
    \hfill
    \begin{minipage}{0.2\textwidth}
        \centering
        \includegraphics[width=\textwidth]{img/logo pic/design&programming/ppt.png}
        \vspace{0.3cm}
        \textbf{PowerPoint}\\
        \small{Mainly used for presentations and slide creation.}
    \end{minipage}
    
    \vspace{0.5cm}
    \noindent
    We used several powerful tools for both programming and design purposes. \textbf{Canva} was utilized for creating visually appealing graphics and designs. \textbf{VSCode} and \textbf{Thonny} were the main programming environments used for coding, while \textbf{Premier Pro} was used for video editing. \textbf{PowerPoint} was mainly used for presentations and slide creation.
\end{figure}

\subsection{3D Design Tools Used}
\begin{figure}[htbp]
    \centering
    \begin{minipage}{0.2\textwidth}
        \centering
        \includegraphics[width=\textwidth]{img/logo pic/3D design/onshape.png}
        \vspace{0.3cm}
        \textbf{OnShape}\\
        \small{Used for designing 3D models with collaboration features.}
    \end{minipage}%
    \hfill
    \begin{minipage}{0.2\textwidth}
        \centering
        \includegraphics[width=\textwidth]{img/logo pic/3D design/studio.io.png}
        \vspace{0.3cm}
        \textbf{Studio.io}\\
        \small{Used for designing 3D models with collaboration features.}
    \end{minipage}%
    \hfill
    \begin{minipage}{0.2\textwidth}
        \centering
        \includegraphics[width=\textwidth]{img/logo pic/3D design/creality.png}
        \vspace{0.3cm}
        \textbf{Creality}\\
        \small{Used for 3D printing.}
    \end{minipage}%
    \hfill
    \begin{minipage}{0.2\textwidth}
        \centering
        \includegraphics[width=\textwidth]{img/logo pic/3D design/blender.png}
        \vspace{0.3cm}
        \textbf{Blender}\\
        \small{Used for advanced 3D modeling and animation.}
    \end{minipage}
    
    \vspace{0.5cm}
    \noindent
     For 3D design, we employed various tools. \textbf{OnShape} and \textbf{Studio.io} helped in designing 3D models with collaboration features. \textbf{Creality} was used for 3D printing, and \textbf{Blender} was employed for advanced 3D modeling and animation.
\end{figure}

\newpage

\subsection{Tools Used}
\begin{figure}[htbp]
    \centering
    \begin{minipage}{0.2\textwidth}
        \centering
        \includegraphics[width=\textwidth]{img/logo pic/tools/Github.png}
        \vspace{0.3cm}
        \textbf{GitHub}\\
        \small{Used for version control of our code.}
    \end{minipage}%
    \hfill
    \begin{minipage}{0.2\textwidth}
        \centering
        \includegraphics[width=\textwidth]{img/logo pic/tools/chatGPT.png}
        \vspace{0.3cm}
        \textbf{ChatGPT}\\
        \small{Assisted in debugging and improving our programming tasks.}
    \end{minipage}%
    \hfill
    \begin{minipage}{0.2\textwidth}
        \centering
        \includegraphics[width=\textwidth]{img/logo pic/tools/Discord.png}
        \vspace{0.3cm}
        \textbf{Discord}\\
        \small{Used for team discussions.}
    \end{minipage}%
    \hfill
    \begin{minipage}{0.2\textwidth}
        \centering
        \includegraphics[width=\textwidth]{img/logo pic/tools/Reverso.png}
        \vspace{0.3cm}
        \textbf{Reverso}\\
        \small{A useful translation tool.}
    \end{minipage}%
    \hfill
    \begin{minipage}{0.2\textwidth}
        \centering
        \includegraphics[width=\textwidth]{img/logo pic/tools/Drive.png}
        \vspace{0.3cm}
        \textbf{Google Drive}\\
        \small{Used for cloud storage and document sharing.}
    \end{minipage}
    
    \vspace{0.5cm}
    \noindent
   For collaboration, version control, and communication, we made use of \textbf{GitHub}, \textbf{ChatGPT}, and \textbf{Discord}. \textbf{GitHub} provided version control for our code, while \textbf{ChatGPT} assisted in debugging and improving our programming tasks. \textbf{Discord} was used for team discussions. \textbf{Reverso} was a useful translation tool, and \textbf{Google Drive} served for cloud storage and document sharing.
\end{figure}



\newpage












%Chapitre2--------------------------------------------------------------------------
\newpage
\chapter{Robot Design}

\vfill 

\begin{center}
    \includegraphics[width=0.6\textwidth]{img/Robot Design/robot design.png}
    \\ 
    
    \vspace{1cm} 

    \includegraphics[width=0.8\textwidth]{img/Robot Design/improvement/base.png} 
    \\ 
\end{center}

\vfill 



\newpage
\section{Our Robot}

In preparation for the FLL Submerged 2025 competition, our team began by thoroughly reviewing the competition rules to understand the key objectives and challenges. We then watched the season teaser to gain insight into the theme and challenges, which inspired our approach to designing and building the robot. To equip ourselves with the necessary skills and knowledge, we explored several online resources, including the Prime Lessons website, which provided us with valuable guidance on both building and coding the robot. Additionally, we watched multiple YouTube tutorials to further enhance our understanding of the technical aspects of the competition. This combination of rule analysis, inspirational content, and resourceful learning helped us lay the foundation for our project, leading to the development of a functional robot that meets the competition’s requirements.



\begin{figure}[h!]
    \centering
    % First row with two images side by side
    \begin{minipage}{0.45\textwidth}
        \centering
        \includegraphics[width=\linewidth]{img//Robot Design//robot game/engineering notebook.png}
       
    \end{minipage}
    \hfill
    \begin{minipage}{0.45\textwidth}
        \centering
        \includegraphics[width=\linewidth]{img//Robot Design//robot game/robot-game-rule}
       
    \end{minipage}
\end{figure}


\begin{figure}[h!]
    \centering
    % First row with two images side by side
    \begin{minipage}{0.45\textwidth}
        \centering
        \qrcode[height=5cm]{https://https://www.youtube.com/watch?v=J5u-2q_K3O0&t}
        \caption{FLL season teaser.}
       
    \end{minipage}
    \hfill
    \begin{minipage}{0.45\textwidth}
        \centering
        \qrcode[height=5cm]{https://primelessons.org/en/}
        \caption{Prime Lessons Website}
       
    \end{minipage}
\end{figure}

\begin{figure}
    \centering
    \includegraphics[width=1\linewidth]{img//Robot Design/ressources.png}
    \caption{Ressources}
    \label{fig:ress}
\end{figure}

\subsection{Hardware}

For our robot design, we had the option to choose from three platforms: LEGO EV3, LEGO Spike Prime, and LEGO Robot Inventor. After careful consideration of the competition’s requirements and the capabilities of each platform, we decided to use the LEGO Robot Inventor set. One of the key reasons for this decision was the wide range of Technic pieces available in the Robot Inventor set, which provided us with more flexibility and precision in designing a robot that could meet the competition’s challenges. The additional Technic components allowed us to create a more robust and versatile robot, capable of handling the tasks efficiently and effectively. This choice enabled us to leverage the strengths of the Robot Inventor system, ensuring that our robot would perform optimally throughout the competition.

\begin{figure}[h!]
    \centering
    % First row with three images side by side
    \begin{minipage}{0.3\textwidth}
        \centering
        \includegraphics[width=\linewidth]{img//Robot Design/ev3.png}
        \caption{EV3 robot}
    \end{minipage}
    \hfill
    \begin{minipage}{0.3\textwidth}
        \centering
        \includegraphics[width=\linewidth]{img/Robot Design/spike-prime.png}
        \caption{Spike prime}
    \end{minipage}
    \hfill
    \begin{minipage}{0.3\textwidth}
        \centering
        \includegraphics[width=\linewidth]{img//Robot Design/inventor.png}
        \caption{Inventor}
    \end{minipage}
\end{figure}

\subsection{Software}

When selecting the programming environment for our robot, we had the option of using either the default block-based coding or the Pybricks firmware, which allows for Python-based programming. After evaluating both options, we chose Pybricks because it offers greater flexibility, efficiency, and control over the robot’s behavior. Unlike block coding, which can be limiting for complex tasks, Pybricks enables us to write more advanced and optimized code, making our robot faster and more responsive. Additionally, Pybricks allows for direct control of motors and sensors with more precise commands, giving us an advantage in completing missions with higher accuracy. This choice also helped us improve our programming skills and better understand real-world coding applications, making our robot more competitive in the FLL Submerged 2025 competition.


\begin{figure}[h!]
    \centering
    % First row with two images side by side
    \begin{minipage}{0.45\textwidth}
        \centering
        \includegraphics[width=0.5\linewidth]{img//Robot Design/code-blocks.png}
        \caption{Spike code-blocks}
       
    \end{minipage}
    \hfill
    \begin{minipage}{0.45\textwidth}
        \centering
        \includegraphics[width=0.5\linewidth]{img//Robot Design/pybricks.png}
        \caption{Prime Lessons Website}
       
    \end{minipage}
\end{figure}

\vspace{1cm}

\section{Robot Game strategy }


%Run1--------------------------------------------------------------------------

\newpage

\subsection{Run 1}
\begin{figure}[h]
    \centering
    \includegraphics[width=1\textwidth]{img/Robot Design/robot game/R1.png}
    \caption{Robot path from the Red area, scoring 95 points, and returning.}
    \label{fig:robot_path}
\end{figure}
\subsubsection{Robot Path Overview}

The robot starts from the Red launch area and follows a predefined path. It takes approximately 19 seconds to reach the scoring zone, where it scores 95 points. After scoring, the robot returns to the Red launch area, completing its task.
\subsubsection{code}
\begin{lstlisting}
def RUN1():
    gyro(True)
    move(-20, 700, 500) 
    turn(60)
    move(-380, 700, 500) 
    turn(-60)
    move(-270, 400, 200) 
    Attachments.A1.diver_open(720)
    turn(90.5)
    gyro(False)
    move(100, 200, 100)  
    gyro(True)
    Attachments.A1.m1_open(1600)
    move(-85, 700, 500)
    turn(40)   
    run_task(Attachments.A1.multitask1())
    turn(80)
    run_task(Attachments.A1.multitask2())
    gyro(False)
\end{lstlisting}






%Run2--------------------------------------------------------------------------
\newpage
\subsection{Run 2}
\begin{figure}[h]
    \centering
    \includegraphics[width=1\textwidth]{img/Robot Design/robot game/R2.png}
    \caption{Robot path from the blue area, scoring 90 points, and returning.}
    \label{fig:robot_path}
\end{figure}
\subsubsection{Robot Path Overview}

The robot starts from the Red launch area and follows a predefined path. It takes approximately 23 seconds to reach the scoring zone, where it scores 90 points. After scoring, the robot returns to the Red launch area, completing its task.
\subsubsection{code}
\begin{lstlisting}
def RUN2_1():
    gyro(True)
    run_task(Attachments.A2.multitask1())
    turn(30)
    move(400, 700, 600)
    turn(60)
    move(210, 800, 800)
    Attachments.A2.fish_open(-670)
    move(-180, 700, 600)
    turn(125)
    move(-180, 700, 600)
    turn(-36)
    move(-130, 900, 800)
    Attachments.A2.diver_open(-1000)
    move(100, 700, 600)
    turn(30)
    gyro(False)
    move(670, 900, 800)
\end{lstlisting}
\begin{lstlisting}
def RUN2_2():
    gyro(True)
    move(-180,800,800)
    Attachments.A2.diver_open(1200)
    move(200,800,800)
    Attachments.A2.diver_open(300)
    gyro(False)
    move(200,800,800)
\end{lstlisting}





%Run3--------------------------------------------------------------------------
\newpage
\subsection{Run 3}
\begin{figure}[h]
    \centering
    \includegraphics[width=1\textwidth]{img/Robot Design/robot game/R3.png}
    \caption{Robot path from the red area, scoring 70 points, returns to the blue launch area.}
    \label{fig:robot_path}
\end{figure}
\subsubsection{Robot Path Overview}

The robot starts from the Red launch area and follows a predefined path. It takes approximately 30 seconds to reach the scoring zone, where it scores 70 points. After scoring, the robot returns to the blue launch area, completing its task.
\subsubsection{code}
\begin{lstlisting}
def RUN3():
    gyro(True)
    move(110,600, 400)
    turn(70)
    move(550, 600, 400)
    turn(-32)
    gyro(False)
    move(114 500, 300)
    gyro(True)
    Attachments.A3.trident_open(-4200)
    Attachments.A3.trident_open(4150)
    move(-110, 700, 600)
    turn(49)
    move(385, 700, 600)
    Attachments.A3.trident_open(-1200)
    move(-230, 500, 300)
    Attachments.A3.trident_open(1200)
    turn(-30)
    move(250, 800, 700)
    turn(50)
    move(850, 800, 750)
    gyro(False)
\end{lstlisting}






%Run4--------------------------------------------------------------------------
\newpage
\subsection{Run 4}
\begin{figure}[h]
    \centering
    \includegraphics[width=1\textwidth]{img/Robot Design/robot game/R4.png}
    \caption{Robot path from the blue area, scoring 60 points, and returns to the blue launch area.}
    \label{fig:robot_path}
\end{figure}
\subsubsection{Robot Path Overview}

The robot starts from the blue launch area and follows a predefined path. It takes approximately 28 seconds to reach the scoring zone, where it scores 60 points. After scoring, the robot returns to the blue launch area, completing its task.
\subsubsection{code}
\begin{lstlisting}
def RUN4():
    gyro(True)
    move(-40,700,500) 
    turn(-60)
    move(-615,700,500)
    turn(33)
    move(-500)
    Attachments.A4.griper(1400)
    move(120, 700, 500)
    turn(-63)
    move(690, 700, 500)
    Attachments.A4.arm(970)
    move(-110, 700, 500)
    turn(90)
    Attachments.A4.arm(-350)
    move(415, 700, 500)
    turn(-45)
    move(50, 700, 500)
    move(-300, 700, 500)
    run_task(Attachments.A4.multitask1())
    turn(-90)
    move(170, 700, 500)
    Attachments.A4.arm(-400)
    turn(110)
    gyro(False)
    move(300, 700, 500)
\end{lstlisting}






%Run5--------------------------------------------------------------------------
\newpage
\subsection{Run 5}
\begin{figure}[h]
    \centering
    \includegraphics[width=1\textwidth]{img/Robot Design/robot game/R5.png}
    \caption{Robot path from the blue area, scoring 80 points, and returns to the blue launch area.}
    \label{fig:robot_path}
\end{figure}
\subsubsection{Robot Path Overview}

The robot starts from the blue launch area and follows a predefined path. It takes approximately 20 seconds to reach the scoring zone, where it scores 80 points. After scoring, the robot returns to the blue launch area, completing its task.
\subsubsection{code}
\begin{lstlisting}
def RUN5():
    gyro(True)
    move(190, 700, 600)
    turn(-25)
    move(618, 700, 600)
    turn(70)
    move(167, 700, 600)
    Attachments.A5.krill_open(-1200)
    Attachments.A5.krill_open(1300)
    move(-175, 700, 600)
    turn(45)
    move(-120, 700, 600)
    Attachments.A5.green_open(2100)
    turn(67)
    run_task(Attachments.A5.multitask1())
\end{lstlisting}






%Run6--------------------------------------------------------------------------
\newpage
\subsection{Run 6}
\begin{figure}[h]
    \centering
    \includegraphics[width=1\textwidth]{img/Robot Design/robot game/R6.png}
    \caption{Robot path from the blue area, scoring 65 points, and returns to the blue launch area.}
    \label{fig:robot_path}
\end{figure}
\subsubsection{Robot Path Overview}

The robot starts from the blue launch area and follows a predefined path. It takes approximately 4 seconds to reach the scoring zone, where it scores 65 points. After scoring, the robot returns to the blue launch area, completing its task.
\subsubsection{code}
\begin{lstlisting}
def RUN6():
    gyro(True)
    move(220,700,500)
    Attachments.A6.bato(800)
    gyro(False)
    move(-250,700,500)
    turn(45)
\end{lstlisting}






%Run7--------------------------------------------------------------------------
\newpage
\subsection{Run 7}
\begin{figure}[h]
    \centering
    \includegraphics[width=1\textwidth]{img/Robot Design/robot game/R7.png}
    \caption{Robot path from the blue area, scoring 80 points, and returns to the blue launch area.}
    \label{fig:robot_path}
\end{figure}
\subsubsection{Robot Path Overview}

The robot starts from the blue launch area and follows a predefined path. It takes approximately 13 seconds to reach the scoring zone, where it scores 80 points. After scoring, the robot returns to the blue launch area, completing its task.
\subsubsection{code}
\begin{lstlisting}
def RUN7():
    gyro(True)
    move(180,700,500) 
    turn(-70)
    move(610,700,500)
    turn(30)
    move(250,700,500)
    turn(-15)
    move(60,700,500)
    move(-80,700,500)
    turn(55)
    gyro(False)
    move(250,700,500)
    gyro(True)
    Attachments.A7.m_10(550)
    wait(1500)
    move(-100,700,500)
    hub.system.shutdown()
\end{lstlisting}


\newpage
\begin{lstlisting}
from RUNS import *

COLOR1 = sensor1.color()    # First sensor detected color
COLOR2 = sensor2.color()    # Second sensor detected color  

hub.display.icon(Icon.HEART)
pressed = []

while True:
    COLOR1 = sensor1.color()    # First sensor detected color
    COLOR2 = sensor2.color()    # Second sensor detected color  

    if COLOR1 == RED and COLOR2 == WHITE:
        pressed = hub.buttons.pressed()
        hub.display.char("1")
        if Button.LEFT in pressed:
            wait(100)
            RUN1()
    elif COLOR1 == WHITE and COLOR2 == RED:
        pressed = hub.buttons.pressed()
        hub.display.char("2")
        if Button.LEFT in pressed:
            wait(100)
            RUN2_1()
        if Button.RIGHT in pressed:
            wait(100)
            RUN2_2()  
    elif COLOR1 == RED and COLOR2 == RED:
        pressed = hub.buttons.pressed()
        hub.display.char("3")
        if Button.LEFT in pressed:
            wait(100)
            RUN3()
    elif COLOR1 == WHITE and COLOR2 == WHITE:
        pressed = hub.buttons.pressed()
        hub.display.char("4")
        if Button.LEFT in pressed:
            wait(100)
            RUN4()
    elif COLOR1 ==BLUE and COLOR2 == BLUE:
        pressed = hub.buttons.pressed()
        hub.display.char("5")
        if Button.LEFT in pressed:
            wait(100)
            RUN5()
    elif COLOR1 == RED and COLOR2 == BLUE:
        pressed = hub.buttons.pressed()
        hub.display.char("6")
        if Button.LEFT in pressed:
            wait(100)
            RUN6()
    elif COLOR1 == WHITE and COLOR2 == BLUE:
        pressed = hub.buttons.pressed()
        hub.display.char("7")
        if Button.LEFT in pressed:
            wait(100)
            RUN7()
\end{lstlisting}
\newpage

\newpage

\begin{center}
    \huge \textbf{\textcolor{orange}{TIMELINE}} \\[0.5cm]
\end{center}
\section{Timeline}
\subsection{First Qualifying Phase}
\vspace{1cm}
\begin{figure}[ht]
    \centering
    \includegraphics[width=1\textwidth]{img/Robot Design/timeline/RG Q1.png}  
    \includegraphics[width=1\textwidth]{img/Robot Design/timeline/RD Q1.png}  
    \label{fig: timeline Q1}
\end{figure}

The preparation for the First Qualifying Phase in robot game and robot design consisted of our initial steps, including learning about the game rules, watching FLL videos, brainstorming strategy, and designing our robot. Below is a breakdown of the tasks we completed week by week:

\begin{enumerate}[1.]
    \item \textbf{Week 1: Initial Research and Setup}
    \begin{itemize}
        \item Learn about the game rules and objectives.
        \item Watch FLL videos to gain insights into robot design and strategies.
        \item Brainstorm possible strategies for the competition.
        \item Begin brainstorming robot design ideas, focusing on the base design.
        \item Set up a GitHub repository to manage our project files and code.
    \end{itemize}

    \item \textbf{Week 2: Strategy and Base Design}
    \begin{itemize}
        \item Choose a strategy to focus on and refine the robot’s design.
        \item Continue building the robot’s base while incorporating improvements.
        \item Start parallel research and development on the robot’s functionality.
        \item Familiarize the team with GitHub for version control.
    \end{itemize}
    
    \item \textbf{Week 3: Prototype Building and Attachment Design}
    \begin{itemize}
        \item Continue building and refining the robot's base structure.
        \item Gather feedback from teammates to improve robot design.
        \item Begin designing and coding attachments for the robot.
        \item Test attachments with the base to ensure compatibility and functionality.
    \end{itemize}

    \item \textbf{Week 4: Attachment Development and Testing}
    \begin{itemize}
        \item Finalize the design and coding for attachments.
        \item Perform parallel testing of attachments and base structure.
        \item Set up a “readme” file and a Bill of Materials (BOM) list for the project.
        \item Continue refining attachments based on test results.
    \end{itemize}

    \item \textbf{Week 5: Final Testing and Adjustments}
    \begin{itemize}
        \item Finalize all attachments and their code.
        \item Perform extensive testing, identify issues, and fix them.
        \item Iteratively improve the robot based on test results.
    \end{itemize}
\end{enumerate}

\subsection{Second Qualifying Phase}

\begin{figure}[ht]
    \centering
    \includegraphics[width=0.6\textwidth]{img/Robot Design/timeline/RG Q2.png}  
    \includegraphics[width=0.6\textwidth]{img/Robot Design/timeline/RD Q2.png}  
    \caption{Timeline Q2}
    \label{fig:timeline}
\end{figure}

In the Second Qualifying Phase, we focused on refining the robot, testing attachments, and addressing any issues identified during previous tests. We also continued to improve our project files, including the GitHub repository.

\begin{enumerate}[1.]
    \item \textbf{Week 6: Icebreaker and Testing}
    \begin{itemize}
        \item Conduct an icebreaker session to strengthen team collaboration.
        \item Identify issues with the robot’s performance and fix them.
        \item Complete the “readme” file on GitHub with updated documentation.
        \item Begin systematic testing to identify weaknesses and areas for improvement.
    \end{itemize}
    
    \item \textbf{Week 7: Finalizing Robot and Testing}
    \begin{itemize}
        \item Continue testing robot performance and iterating on designs.
        \item Make final adjustments based on test results.
        \item Prepare and complete all necessary documentation on GitHub.
    \end{itemize}
\end{enumerate}

\newpage
\begin{center}
    \huge \textbf{\textcolor{orange}{TASK MANAGEMENT}} \\[0.5cm]
\end{center}
\section{Tasks}
 The division of tasks was carefully planned to ensure efficiency and clarity throughout the project. We utilized several criteria to manage and assign tasks, including deadlines, importance, the availability of team members (whether tasks were done online or in-person), and status updates. Each task was broken down into manageable pieces, allowing for smoother execution and parallel progress. 


We used a collaborative approach, ensuring that each member contributed to both the design and development phases of the robot, as well as the testing and iteration stages. This enabled us to leverage the strengths of every team member and efficiently work on different components of the robot. 

\begin{figure}[ht]
    \centering
    \includegraphics[width=1\textwidth]{img/Robot Design/tasks/tasks.png} 
    \caption{Task Assignment for Robot Design}
\end{figure}

In this section, we describe how tasks were distributed among team members for the robot design process.

\subsection{Task Assignment}
The tasks were divided among the team based on both design and coding responsibilities, with each pair of team members focusing on specific attachments. Here's an overview of how the tasks were assigned:

\begin{itemize}
    \item \textbf{Mortada and Salman:} Worked on Attachment 1 and Attachment 2
    \item \textbf{Walid and Ahmed:} Worked on Attachment 6
    \item \textbf{Omar and Salma Yacoubi:} Worked on Attachment 5
    \item \textbf{Rayan and Mohammed:} Worked on Attachment 3
    \item \textbf{Mariam and Salma:} Worked on Attachment 4
\end{itemize}



\newpage
\begin{center}
    \huge \textbf{\textcolor{orange}{CODING}} \\[0.5cm]
\end{center}
\section{Coding}


\subsection{Introduction to the Coding Process}
For our robot, we used \textbf{MicroPython} as the programming language, along with the \textbf{Pybricks} framework. Pybricks is specifically designed for programming LEGO robots and provides an efficient way to control motors, sensors, and other hardware components. \newline
To manage our code and collaborate effectively, we utilized \textbf{GitHub}, where we stored our files, tracked changes, and worked on different versions of our program.

\begin{center}
  \includegraphics[width=0.6\textwidth]{img/Robot Design/CODING/Coding process.png}
\end{center}
\subsection{File Structure}
Our project was structured into three main Python files:
\begin{itemize}
    \item \textbf{main.py}: This is the central file that calls functions from other files and controls the robot's overall execution.
    \item \textbf{attachments.py}: This file defines the various attachments and mechanisms used in the robot.
    \item \textbf{runs.py}: This file contains predefined sequences of movements and operations for different competition tasks.
\end{itemize} 
\begin{center}
  \includegraphics[width=0.3\textwidth]{img/Robot Design/CODING/structure.png}
\end{center}
Each file plays a crucial role in maintaining a well-organized and modular code structure.

\subsection{Libraries Used}

We imported essential libraries from Pybricks, including:
\begin{itemize}
    \item \texttt{pybricks.hubs}: For controlling the hub.
    \item \texttt{pybricks.pupdevices}: For handling motors and sensors.
    \item \texttt{pybricks.parameters}: For defining ports, directions, and other configurations.
    \item \texttt{pybricks.robotics}: For drivebase control.
    \item \texttt{pybricks.tools}: For using wait times, tasks, and multitasking features.
\end{itemize}
\begin{center}
  \includegraphics[width=0.6\textwidth]{img/Robot Design/CODING/libraries.png}
\end{center}


\subsection{Defining Ports and Devices}
To ensure proper hardware configuration, we defined all the motors and sensors with their respective ports:
\begin{itemize}
    \item Motors: Left, right, front, and back motors assigned to specific ports.
    \item Sensors: Color sensors assigned to different ports.
    \item Drivebase: A combination of motors for overall movement control.
    
\end{itemize}
\vspace{1cm}
\begin{center}
  \includegraphics[width=0.6\textwidth]{img/Robot Design/CODING/ports and devices.png}
\end{center}
\vspace{1cm}


\subsection{Essential Functions}
We created functions for fundamental robot actions:
\begin{itemize}
    \item \textbf{move(distance, speed, acceleration)}: Moves the robot forward or backward at a specified speed and acceleration.
    \item \textbf{turn(rotation)}: Rotates the robot by a certain angle.
\end{itemize}
These functions ensure smooth and precise control of the robot's movement.

\vspace{1cm}
\begin{center}
  \includegraphics[width=0.6\textwidth]{img/Robot Design/CODING/essential functions.png}
\end{center}
\vspace{8cm}
\subsection{Each Attachment Function}
For each attachment, we defined specific functions in the \texttt{attachments.py} file:
\begin{itemize}
    \item \textbf{m1\_open(angle)}: Controls the back motor to run at a specified angle.
    \item \textbf{diver\_open(angle)}: Moves the front motor to a specific angle.
\end{itemize}
These functions allow modular control of different robot mechanisms.
\vspace{1cm}
\begin{center}
  \includegraphics[width=0.6\textwidth]{img/Robot Design/CODING/attachement functions.png}
\end{center}


\subsection{Auto Detection}
Our robot is equipped with an automatic detection system using a color sensor to identify different attachments. This feature allows the robot to recognize which attachment is currently in use and execute the corresponding code accordingly.

\begin{itemize}
    \item 	\textbf{Color Sensor for Attachment Recognition:} The color sensor scans the color of the attachment placed on the robot.
    \item 	\textbf{Automatic Code Selection:} Based on the detected color, the robot automatically selects and runs the pre-programmed code specific to that attachment.
    \item 	\textbf{Efficiency and Precision:} This system eliminates the need for manual selection, reducing errors and improving efficiency during operation.
\end{itemize}

\vspace{0.5cm}
\begin{figure}[h!]
    \centering
    \includegraphics[width=0.75\linewidth]{img//Robot Design/auto-detect.png}
    \caption{Auto-Detection logic}
    \label{fig:auto-detect}
\end{figure}
\newpage
\subsection{Sensors Used}

Our robot utilizes two primary sensors to enhance functionality and precision:
\vspace{1cm}
\begin{center}
  \includegraphics[width=0.6\textwidth]{img/Robot Design/CODING/sensors.png}
\end{center}

\subsection{Gyroscope Sensor}
The gyroscope sensor is used for orientation and balance control. It helps in:

\begin{itemize}
    \item 	\textbf{Maintaining Stability:} Ensuring the robot moves in a straight line without deviation.
    \item 	\textbf{Accurate Turns:} Assisting in precise angle measurements for smooth and controlled turns.
    \item 	\textbf{Reducing Drift:} Minimizing errors caused by external factors like friction or minor obstacles.
\end{itemize}

\subsection{Color Sensor}
The color sensor is a crucial component for multiple functions in the robot:

\begin{itemize}
    \item 	\textbf{Attachment Detection:} Recognizing the color of the attachment to trigger the corresponding code.
    \item 	\textbf{Line Following:} Assisting in navigation by detecting different colors on the surface.
    \item 	\textbf{Object Identification:} Differentiating between objects based on color variations.

\end{itemize}

\newpage
\begin{center}
    \huge \textbf{\textcolor{orange}{ENGINEERING DESIGN PROCESS}} \\[0.5cm]
\end{center}
\section{Engineering Design Process}
The engineering design process helped us build and refine our robot to succeed in the robot game.
\vspace{1cm}
\begin{center}
  \includegraphics[width=0.6\textwidth]{img/Robot Design/EDP2.png}
\end{center}

\subsection{1. Define the Problem}
We identified the goal: to build a robot capable of navigating the game environment and completing tasks with precision and efficiency.

\subsection{2. Research and Brainstorming}
We researched robot designs, mechanisms, and strategies, then brainstormed solutions that fit the game requirements and constraints.

\subsection{3. Develop Solutions}
We chose a design, selected the necessary components (motors, sensors, controllers), and sketched the initial plan.

\subsection{4. Build and Test}
We built the prototype, tested its performance, and identified areas for improvement (speed, maneuverability, and precision).

\subsection{5. Iterate and Improve}
Based on test results, we made adjustments to the robot's design and programming for better performance.

\subsection{6. Finalize the Design}
After refining the robot, we ensured it could effectively complete the tasks in the game.


\newpage
\begin{center}
    \huge \textbf{\textcolor{orange}{ATTEMPTS AND TESTING}} \\[0.5cm]
\end{center}
\section{attempts and testing}
\vspace{0.5cm}
\begin{figure}[h!]
    \centering
    \includegraphics[width=1\linewidth]{img/Robot Design/robot game/attempts/attempts.png}
    
\end{figure}
To ensure the reliability and efficiency of our robot, we conducted a series of tests using a structured approach. We created a Google Sheet table to document the performance of each attachment, across 20 attempts. 


\vspace{0.5cm}
\begin{figure}[h!]
    \centering
    \includegraphics[width=1\linewidth]{img/Robot Design/robot game/attempts/graph.png}
    
\end{figure}
By recording the \textbf{Sucess rate} of each attempt, we were able to analyze patterns, identify weaknesses, and refine our design. Our final results showed an average success rate of \textbf{76\%}, which provided valuable insights into our robot’s consistency and areas for improvement


This technique allowed us to:
\begin{itemize}
    \item Quantify our robot's performance with real data.
    \item Identify inconsistencies and failures.
    \item Improve our design based on measurable results rather than assumptions.
    
\end{itemize}


\newpage
\begin{center}
    \huge \textbf{\textcolor{orange}{ROBOT IMPROVEMENT}} \\[0.5cm]
\end{center}
\section{robot improvements}
\vspace{0.5cm}
\begin{figure}[h!]
    \centering
    \includegraphics[width=0.75\linewidth]{img/Robot Design/improvement/improvement.png}
\end{figure}

Our 2025 robot design features several key improvements over last year's model, making it more efficient and better suited for the competition. 

\subsection{Increased Height}
Last year’s robot was too low, causing it to scratch the ground during movement. This year, we \textbf{raised the robot’s height} to prevent friction and improve navigation.

\subsection{Optimized Motor Placement}
Last year, both motors were placed in the front, limiting the robot’s efficiency. For this current robot, we repositioned one motor to the opposite side, allowing the robot to complete \textbf{two missions simultaneously} instead of one.

\subsection{Reduced Weight}
The previous design was heavier, making movements slower and less precise. This year, we \textbf{reduced the weight}, increasing the robot’s speed and maneuverability.

\subsection{Embedded Motor Design}
In last year’s design, the motor was lifted up, affecting stability. For 2025, we \textbf{integrated the motor inside the structure}, improving balance and overall performance.

These improvements helped us create a more stable and efficient robot for this year’s challenges.


\newpage
\begin{center}
    \huge \textbf{\textcolor{orange}{DEVELOPING STEPS}} \\[0.5cm]
\end{center}
\section{Developing steps}


In the development process of our robot and attachements we focused on a particular aspect of the design, from enhancing the mechanical structure to optimizing its performance for real-world applications. The following sections outline these steps and improvements, referencing key attachments, including Attachment 6, to illustrate the changes and testing results.


\subsection*{Phase 1: Initial Passive Mechanism Design}


In the first phase, we developed a passive mechanism as a baseline for testing. While functional, it lacked adaptability and active control, primarily assessing movement feasibility.
\begin{figure}[h!]
    \centering
        \includegraphics[width=0.5\textwidth]{img/Robot Design/improvement/phase1.png}
        
    \end{figure}
    
\subsection*{Phase 2: Transition to Active Mechanism}

In the second phase, we adopted an active mechanism for greater control and responsiveness. However, the added components increased weight, reducing efficiency.
\begin{figure}[h!]
    \centering
        \includegraphics[width=0.5\textwidth]{img/Robot Design/improvement/phase2.png}
        
    \end{figure}
\newpage    
\subsection*{Phase 3: Weight Reduction and Performance Optimization}

In Phase 3, we reduced weight while preserving the active mechanism's benefits. Through testing and design refinements, we created a lighter, more efficient, and agile robot without compromising functionality.
\begin{figure}[h!]
    \centering
        \includegraphics[width=0.5\textwidth]{img/Robot Design/improvement/phase3.png}
        
    \end{figure}




\newpage
\begin{center}
    \huge \textbf{\textcolor{orange}{PROUDEST PART OF OUR ROBOT DESIGN}} \\[0.5cm]
\end{center}
\section{Proudest Part of Our Robot Design}
\subsection{Robot Base}
\begin{minipage}{0.6\textwidth}
    The base of our robot is one of the most refined aspects of our design. We focused on optimizing efficiency, functionality, and performance. Key features include:

    \begin{itemize}
        \item \textbf{Dual Motor System:} 
        
        By placing two motors on different sides, our robot efficiently handles two missions simultaneously, increasing productivity and speed during runs.
        
        \item \textbf{Lightweight and Optimal Design:}
        
        We carefully engineered the base to be lightweight, ensuring smooth movement without unnecessary bulk.
        \item \textbf{Ground-Friendly Structure:} Our design ensures that the robot base does not scratch the ground, preserving both the field surface and the robot’s durability.
        \item \textbf{Refined Engineering:} Countless hours were dedicated to designing and optimizing the base to ensure it performs at peak efficiency without unnecessary complexity.
    \end{itemize}
\end{minipage}%
\hfill
\begin{minipage}{0.35\textwidth}
    \centering
    \includegraphics[width=\textwidth]{img/Robot Design/improvement/base.png}
\end{minipage}



\subsection{Attachment Base}
\begin{minipage}{0.6\textwidth}
    Another key aspect of our design is the attachment base, which enhances automation and adaptability. Key features include:

    \begin{itemize}
        \item \textbf{Integrated Color Sensors:} 
        
        The attachment base is equipped with color sensors, allowing the robot to recognize different runs based on the attachment’s base color, enabling seamless automation.
        
        \item \textbf{Efficient Run Identification:} 
        
        With color recognition, the robot quickly adapts to different tasks without manual intervention, improving accuracy and efficiency.
        
        \item \textbf{Stable and Reliable Design:} 
        
        The attachment base is engineered for precision, ensuring a secure connection and stable performance during operations.
    \end{itemize}
\end{minipage}%
\hfill
\begin{minipage}{0.35\textwidth}
    \centering
    \includegraphics[width=\textwidth]{img/Robot Design/improvement/base attachements.png}
\end{minipage}





%Chapitre3--------------------------------------------------------------------------
\newpage

\chapter{Core Values}
\vfill 
\begin{center}
    \includegraphics[width=0.7\textwidth]{img/corevalues.png}
    \\ 
    \vspace{1cm} 

    \includegraphics[width=0.5\textwidth]{img/profil pics/fll seniors.png} 
    \\ 
\end{center}
\vfill 

\newpage

\begin{figure}[H]
   
    \begin{tabular}{m{0.2\textwidth} m{0.75\textwidth}}
        \includegraphics[width=\linewidth]{img/profil pics/mortada.jpg} & 
        \textit We are stronger when we work together. Our team has weekly brainstorming sessions where everyone’s ideas matter.
    \end{tabular}
\end{figure}



\begin{figure}[H]
   
    \begin{tabular}{m{0.2\textwidth} m{0.75\textwidth}}
        \includegraphics[width=\linewidth]{img/profil pics/salman.jpg} & 
        \textit Our project is designed to make a difference, and we’re proud to contribute to solving real-world problems.
    \end{tabular}
\end{figure}


\begin{figure}[H]
   
    \begin{tabular}{m{0.2\textwidth} m{0.75\textwidth}}
        \includegraphics[width=\linewidth]{img/profil pics/mariam.jpg} & 
        \textit We respect and embrace everyone’s differences. Our team includes members with different strengths, and that’s our superpower!
    \end{tabular}
\end{figure}



\begin{figure}[H]
   
    \begin{tabular}{m{0.2\textwidth} m{0.75\textwidth}}
        \includegraphics[width=\linewidth]{img/profil pics/samadi.jpg} & 
        \textit By sharing tasks and supporting each other, we’ve achieved so much more as a team.
    \end{tabular}
\end{figure}

\newpage


\begin{figure}[H]
   
    \begin{tabular}{m{0.2\textwidth} m{0.75\textwidth}}
        \includegraphics[width=\linewidth]{img/profil pics/yaakoubi.jpg} & 
        \textit We apply what we learn to improve the world. Our innovation project focuses on [briefly mention the goal of your FLL innovation project
    \end{tabular}
\end{figure}




\begin{figure}[H]
   
    \begin{tabular}{m{0.2\textwidth} m{0.75\textwidth}}
        \includegraphics[width=\linewidth]{img/profil pics/rayane.jpg} & 
        \textit Every day is a chance to discover something new, whether it’s a coding solution or a new robot design.
    \end{tabular}
\end{figure}




\begin{figure}[H]
   
    \begin{tabular}{m{0.2\textwidth} m{0.75\textwidth}}
        \includegraphics[width=\linewidth]{img/profil pics/omar.jpg} & 
        \textit We are stronger when we work together. Our team has weekly brainstorming sessions where everyone’s ideas matter.
    \end{tabular}
\end{figure}


\begin{figure}[H]
   
    \begin{tabular}{m{0.2\textwidth} m{0.75\textwidth}}
        \includegraphics[width=\linewidth]{img/profil pics/walid.jpg} & 
        \textit We celebrate our progress and have fun while learning. Whether we’re testing our robot or working on the project, we always enjoy ourselves.
    \end{tabular}
\end{figure}


\begin{figure}[H]
   
    \begin{tabular}{m{0.2\textwidth} m{0.75\textwidth}}
        \includegraphics[width=\linewidth]{img/profil pics/ahmed.jpg} & 
        \textit We love exploring new skills and ideas. This year, we learned about [mention a new skill or tech used in your project
    \end{tabular}
\end{figure}



\begin{figure}[H]
   
    \begin{tabular}{m{0.2\textwidth} m{0.75\textwidth}}
        \includegraphics[width=\linewidth]{img/profil pics/salma.jpg} & 
        \textit Each one of us contributes unique ideas, and that’s what makes our team dynamic and inclusive.
    \end{tabular}
\end{figure}
